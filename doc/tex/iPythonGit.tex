http://www.astro.washington.edu/users/vanderplas/Astr599/notebooks/07_GitIntro.html


This notebook was put together by Jake Vanderplas for UW's Astro 599 course. Source and license info is on GitHub.
Version control for fun and profit:

the tool you didn't know you needed

Sources of this material:

This tutorial is adapted from "Version Control for Fun and Profit" by Fernando Perez
For an excellent list of Git resources for scientists, see Fernando's Page.
Fernando's original notebook specifically mentions two references he drew from:
"Git for Scientists: A Tutorial" by John McDonnell
Emanuele Olivetti's lecture notes and exercises from the G-Node summer school on Advanced Scientific Programming in Python.
Via Fernando, some of the images below are copied from the Pro Git book
Also see J.R. Johansson's tutorial on version control, part of his excellent series Lectures on Scientific Computing with Python
What is Version Control?

From Wikipedia:

�Revision control, also known as version control, source control or software configuration management (SCM), is the management of changes to documents, programs, and other information stored as computer files.�
Reproducibility?

Tracking and recreating every step of your work
In the software world: it's called Version Control!
What do (good) version control tools give you?
Peace of mind (backups)
Freedom (exploratory branching)
Collaboration (synchronization)
Git is an enabling technology: Use version control for everything

Paper writing (never get paper_v5_jake_final_oct22_9.tex by email again!)
Grant writing
Everyday research
Teaching (never accept an emailed homework assignment again!)
Code management
Personal website history tracking
The plan for this tutorial

Overview of Git key concepts
Hands-on work with Git
5 "stages" of using Git:
Local, single-user, linear workflow
Single local user, branching
Using remotes as a single user
Remotes for collaborating in a small team
Full-contact github: distributed collaboration with large teams
High level picture: overview of key concepts

The commit: a snapshot of work at a point in time

Credit: ProGit book, by Scott Chacon, CC License.
Looking at my current directory:

In [1]:
!ls
00_intro.ipynb                    myfile.html
01_basic_training.ipynb           myfile.py
02_advanced_data_structures.ipynb myfile.pyc
03_IPython_intro.ipynb            mymodule.py
04_Functions_and_modules.ipynb    mymodule.pyc
05_NumpyIntro.ipynb               mymodule2.py
05_Trapezoid_Solution.ipynb       mymodule2.pyc
06_Denoise_Solution.ipynb         number_game.py
06_MatplotlibIntro.ipynb          style.css
07_GitIntro.ipynb                 test.npy
README.txt                        test.npz
images                            test.out
modfun.py                         tmp.py~

A repository: a group of linked commits

Note: these form a Directed Acyclic Graph (DAG), with nodes identified by their hash.
A hash: a fingerprint of the content of each commit and its parent
In [2]:
import sha

# Our first commit
data1 = 'This is the start of my paper2.'
meta1 = 'date: 1/1/12'
hash1 = sha.sha(data1 + meta1).hexdigest()
print 'Hash:', hash1
Hash: 7bb695b77966e27cfaebfa59e27a0b91f1d33813

In [3]:
# Our second commit, linked to the first
data2 = 'Some more text in my paper...'
meta2 = 'date: 1/2/12'
# Note we add the parent hash here!
hash2 = sha.sha(data2 + meta2 + hash1).hexdigest()
print 'Hash:', hash2
Hash: 543da8bac9f643ba5611897b192a16dea42d2ab7

And this is pretty much the essence of Git!
First: Configuring Git

The minimal amount of configuration for git to work without pestering you is to tell it who you are. All the commands here modify the .gitconfig file in your home directory.
Modify these before running them:
In [2]:
%%bash
git config --global user.name "Jake Vanderplas"
git config --global user.email "vanderplas@astro.washington.edu"
Other settings

Change how you will edit text files (it will often ask you to edit messages and other information, and thus wants to know how you like to edit your files):
In [3]:
%%bash
# Put here your preferred editor. If this is not set, git will honor
# the $EDITOR environment variable
git config --global core.editor /usr/bin/nano  # my preferred editor

# On Windows Notepad will do in a pinch,
#   I recommend Notepad++ as a free alternative
# On the mac, you can set nano or emacs as a basic option
In [4]:
%%bash
# And while we're at it, we also turn on the use of color, which is very useful
git config --global color.ui "auto"
Password memory

Set git to use the credential memory cache so we don't have to retype passwords too frequently. On Linux, you should run the following (note that this requires git version 1.7.10 or newer):
In [6]:
%%bash
git config --global credential.helper cache
# Set the cache to timeout after 2 hours (setting is in seconds)
git config --global credential.helper 'cache --timeout=7200'
Github offers in its help pages instructions on how to configure the credentials helper for Mac OSX and Windows.
Double-checking the result:

In [7]:
!cat ~/.gitconfig
[user]
	name = Jake Vanderplas
	email = vanderplas@astro.washington.edu
[core]
	editor = /usr/bin/nano
[color]
	ui = auto
[credential]
	helper = cache --timeout=7200

Stage 1: Local, single-user, linear workflow

Simply type git to see a full list of all the 'core' commands. We'll now go through most of these via small practical exercises:
In [80]:
!git
usage: git [--version] [--exec-path[=<path>]] [--html-path] [--man-path] [--info-path]
           [-p|--paginate|--no-pager] [--no-replace-objects] [--bare]
           [--git-dir=<path>] [--work-tree=<path>] [--namespace=<name>]
           [-c name=value] [--help]
           <command> [<args>]

The most commonly used git commands are:
   add        Add file contents to the index
   bisect     Find by binary search the change that introduced a bug
   branch     List, create, or delete branches
   checkout   Checkout a branch or paths to the working tree
   clone      Clone a repository into a new directory
   commit     Record changes to the repository
   diff       Show changes between commits, commit and working tree, etc
   fetch      Download objects and refs from another repository
   grep       Print lines matching a pattern
   init       Create an empty git repository or reinitialize an existing one
   log        Show commit logs
   merge      Join two or more development histories together
   mv         Move or rename a file, a directory, or a symlink
   pull       Fetch from and merge with another repository or a local branch
   push       Update remote refs along with associated objects
   rebase     Forward-port local commits to the updated upstream head
   reset      Reset current HEAD to the specified state
   rm         Remove files from the working tree and from the index
   show       Show various types of objects
   status     Show the working tree status
   tag        Create, list, delete or verify a tag object signed with GPG

See 'git help <command>' for more information on a specific command.

git init: create an empty repository

In [8]:
%%bash
rm -rf test
git init test
Initialized empty Git repository in /Users/jakevdp/Opensource/2013_fall_ASTR599/notebooks/test/.git/

Note: all these cells below are meant to be run by you in a terminal where you change once to the test directory and continue working there.
Since we are putting all of them here in a single notebook for the purposes of the tutorial, they will all be prepended with the first two lines:
%%bash
cd test
that tell IPython to do that each time. But you should ignore those two lines and type the rest of each cell yourself in your terminal.
Let's look at what git did:
In [9]:
%%bash
cd test

ls
In [40]:
%%bash
cd test

ls -la
total 12
drwxr-xr-x 3 fperez wavelet 4096 Feb 14 00:57 .
drwxr-xr-x 7 fperez wavelet 4096 Feb 14 00:57 ..
drwxr-xr-x 7 fperez wavelet 4096 Feb 14 00:57 .git

In [41]:
%%bash
cd test

ls -l .git
total 32
drwxr-xr-x 2 fperez wavelet 4096 Feb 14 00:57 branches
-rw-r--r-- 1 fperez wavelet   92 Feb 14 00:57 config
-rw-r--r-- 1 fperez wavelet   73 Feb 14 00:57 description
-rw-r--r-- 1 fperez wavelet   23 Feb 14 00:57 HEAD
drwxr-xr-x 2 fperez wavelet 4096 Feb 14 00:57 hooks
drwxr-xr-x 2 fperez wavelet 4096 Feb 14 00:57 info
drwxr-xr-x 4 fperez wavelet 4096 Feb 14 00:57 objects
drwxr-xr-x 4 fperez wavelet 4096 Feb 14 00:57 refs

Now let's edit our first file in the test directory with a text editor... I'm doing it programatically here for automation purposes, but you'd normally be editing by hand
In [11]:
%%bash
cd test

echo "My first bit of text" > file1.txt
In [13]:
%%bash
cd test

ls -al
total 8
drwxrwxrwx   4 jakevdp  staff   136 Sep 12 10:05 .
drwxr-xr-x  30 jakevdp  staff  1020 Sep 12 10:03 ..
drwxrwxrwx   9 jakevdp  staff   306 Sep 12 10:03 .git
-rw-rw-rw-   1 jakevdp  staff    21 Sep 12 10:05 file1.txt

git add: tell git about this new file

In [14]:
%%bash
cd test

git add file1.txt
We can now ask git about what happened with status:
In [15]:
%%bash
cd test

git status
# On branch master
#
# Initial commit
#
# Changes to be committed:
#   (use "git rm --cached <file>..." to unstage)
#
#	new file:   file1.txt
#

git commit: permanently record our changes in git's database

For now, we are always going to call git commit either with the -a option or with specific filenames (git commit file1 file2...).
This delays the discussion of an aspect of git called the index (often referred to also as the 'staging area') that we will cover later. Most everyday work in regular scientific practice doesn't require understanding the extra moving parts that the index involves, so on a first round we'll bypass it. Later on we will discuss how to use it to achieve more fine-grained control of what and how git records our actions.
In [16]:
%%bash
cd test

git commit -a -m"This is our first commit"
[master (root-commit) 609a459] This is our first commit
 1 file changed, 1 insertion(+)
 create mode 100644 file1.txt

In the commit above, we used the -m flag to specify a message at the command line.
If we don't do that, git will open the editor we specified in our configuration above and require that we enter a message.
By default, git refuses to record changes that don't have a message to go along with them (though you can obviously 'cheat' by using an empty or meaningless string: git only tries to facilitate best practices, it's not your nanny).
git log: what has been committed so far

In [17]:
%%bash
cd test

git log
commit 609a45916899c56d1e3b8fe021b907f58845e75e
Author: Jake Vanderplas <vanderplas@astro.washington.edu>
Date:   Thu Sep 12 10:06:55 2013 -0700

    This is our first commit

git diff: what have I changed?

Let's do a little bit more work... Again, in practice you'll be editing the files by hand, here we do it via shell commands for the sake of automation (and therefore the reproducibility of this tutorial!)
In [18]:
%%bash
cd test

echo "And now some more text..." >> file1.txt
And now we can ask git what is different:
In [19]:
%%bash
cd test

git diff
diff --git a/file1.txt b/file1.txt
index ce645c7..4baa979 100644
--- a/file1.txt
+++ b/file1.txt
@@ -1 +1,2 @@
 My first bit of text
+And now some more text...

The cycle of git virtue: work, commit, work, commit, ...

In [20]:
%%bash
cd test

git commit -a -m"I have made great progress on this critical matter."
[master 43ed8dd] I have made great progress on this critical matter.
 1 file changed, 1 insertion(+)

git log revisited

First, let's see what the log shows us now:
In [21]:
%%bash
cd test

git log
commit 43ed8dd431bde1d511934526da8414113965178b
Author: Jake Vanderplas <vanderplas@astro.washington.edu>
Date:   Thu Sep 12 10:09:41 2013 -0700

    I have made great progress on this critical matter.

commit 609a45916899c56d1e3b8fe021b907f58845e75e
Author: Jake Vanderplas <vanderplas@astro.washington.edu>
Date:   Thu Sep 12 10:06:55 2013 -0700

    This is our first commit

Sometimes it's handy to see a very summarized version of the log:
In [51]:
%%bash
cd test

git log --oneline --topo-order --graph
* 2d29a7b I have made great progress on this critical matter!
* 679f246 This is our first commit

Defining an alias

Git supports aliases: new names given to command combinations. Let's make this handy shortlog an alias, so we only have to type git slog and see this compact log:
In [22]:
%%bash
cd test

# We create our alias (this saves it in git's permanent configuration file):
git config --global alias.slog "log --oneline --topo-order --graph"

# And now we can use it
git slog
* 43ed8dd I have made great progress on this critical matter.
* 609a459 This is our first commit

git mv and rm: moving and removing files

While git add is used to add fils to the list git tracks, we must also tell it if we want their names to change or for it to stop tracking them. In familiar Unix fashion, the mv and rm git commands do precisely this:
In [23]:
%%bash
cd test

git mv file1.txt file-newname.txt
git status
# On branch master
# Changes to be committed:
#   (use "git reset HEAD <file>..." to unstage)
#
#	renamed:    file1.txt -> file-newname.txt
#

Note that these changes must be committed too, to become permanent! In git's world, until something hasn't been committed, it isn't permanently recorded anywhere.
In [24]:
%%bash
cd test

git commit -a -m"I like this new name better"
echo "Let's look at the log again:"
git slog
[master 110c1bb] I like this new name better
 1 file changed, 0 insertions(+), 0 deletions(-)
 rename file1.txt => file-newname.txt (100%)
Let's look at the log again:
* 110c1bb I like this new name better
* 43ed8dd I have made great progress on this critical matter.
* 609a459 This is our first commit

And git rm works in a similar fashion.
Exercise

Add a new file file2.txt, commit it, make some changes to it, commit them again, and then remove it (and don't forget to commit this last step!).
2. Single Local user, branching

What is a branch? Simply a label for the 'current' commit in a sequence of ongoing commits:

Mulitple Branches

There can be multiple branches alive at any point in time; the working directory is the state of a special pointer called HEAD. In this example there are two branches, master and testing, and testing is the currently active branch since it's what HEAD points to:

Once new commits are made on a branch, HEAD and the branch label move with the new commits:

This allows the history of both branches to diverge:

But based on this graph structure, git can compute the necessary information to merge the divergent branches back and continue with a unified line of development:

Branching Example

Let's now illustrate all of this with a concrete example. Let's get our bearings first:
In [25]:
%%bash
cd test

git status
ls
# On branch master
nothing to commit, working directory clean
file-newname.txt

We are now going to try two different routes of development: on the master branch we will add one file and on the experiment branch, which we will create, we will add a different one. We will then merge the experimental branch into master.
In [26]:
%%bash
cd test

git branch experiment
git checkout experiment
Switched to branch 'experiment'

In [27]:
%%bash
cd test

echo "Some crazy idea" > experiment.txt
git add experiment.txt
git commit -a -m"Trying something new"
git slog
[experiment 2ac278c] Trying something new
 1 file changed, 1 insertion(+)
 create mode 100644 experiment.txt
* 2ac278c Trying something new
* 110c1bb I like this new name better
* 43ed8dd I have made great progress on this critical matter.
* 609a459 This is our first commit

In [28]:
%%bash
cd test

git checkout master
git slog
* 110c1bb I like this new name better
* 43ed8dd I have made great progress on this critical matter.
* 609a459 This is our first commit

Switched to branch 'master'

In [29]:
%%bash
cd test

echo "All the while, more work goes on in master..." >> file-newname.txt
git commit -a -m"The mainline keeps moving"
git slog
[master 0c5002e] The mainline keeps moving
 1 file changed, 1 insertion(+)
* 0c5002e The mainline keeps moving
* 110c1bb I like this new name better
* 43ed8dd I have made great progress on this critical matter.
* 609a459 This is our first commit

In [30]:
%%bash
cd test

ls
file-newname.txt

In [31]:
%%bash
cd test

git merge experiment
git slog
Merge made by the 'recursive' strategy.
 experiment.txt | 1 +
 1 file changed, 1 insertion(+)
 create mode 100644 experiment.txt
*   3471d04 Merge branch 'experiment'
|\  
| * 2ac278c Trying something new
* | 0c5002e The mainline keeps moving
|/  
* 110c1bb I like this new name better
* 43ed8dd I have made great progress on this critical matter.
* 609a459 This is our first commit

3. Using remotes as a single user

We are now going to introduce the concept of a remote repository: a pointer to another copy of the repository that lives on a different location. This can be simply a different path on the filesystem or a server on the internet.
For this discussion, we'll be using remotes hosted on the GitHub.com service, but you can equally use other services like BitBucket or Gitorious as well as host your own.
If you don't have a Github account, take a moment now to sign up
git remote: view/modify remote repositories

In [61]:
%%bash
cd test

ls
echo "Let's see if we have any remote repositories here:"
git remote -v
experiment.txt
file-newname.txt
Let's see if we have any remote repositories here:

Since the above cell didn't produce any output after the git remote -v call, it means we have no remote repositories configured.
Configuring a remote

Log into GitHub, go to the new repository page and make a repository called test.
Do not check the box that says Initialize this repository with a README, since we already have an existing repository here. That option is useful when you're starting first at Github and don't have a repo made already on a local computer.
We can now follow the instructions from the next page:
In [32]:
%%bash
cd test

git remote add origin https://github.com/jakevdp/test.git
Process is terminated.

Let's see the remote situation again:
In [67]:
%%bash
cd test

git remote -v
origin	https://github.com/fperez/test.git (fetch)
origin	https://github.com/fperez/test.git (push)

Pushing changes to a remote repository

Now push the master branch to the remote named origin:
In []:
%%bash
cd test

git push origin master
We can now see this repository publicly on github.
Using Git to Sync Work

Let's see how this can be useful for backup and syncing work between two different computers. I'll simulate a 2nd computer by working in a different directory...
In [34]:
%%bash
# Here I clone my 'test' repo but with a different name, test2, to simulate a 2nd computer
git clone https://github.com/jakevdp/test.git test2
cd test2
pwd
git remote -v
Cloning into 'test2'...
Checking connectivity... done
/Users/jakevdp/Opensource/2013_fall_ASTR599/notebooks/test2
origin	https://github.com/jakevdp/test.git (fetch)
origin	https://github.com/jakevdp/test.git (push)

Let's now make some changes in one 'computer' and synchronize them on the second.
In [35]:
%%bash
cd test2  # working on computer #2

echo "More new content on my experiment" >> experiment.txt
git commit -a -m"More work, on machine #2"
[master 8852251] More work, on machine #2
 1 file changed, 1 insertion(+)

Now we put this new work up on the github server so it's available from the internet
In [39]:
%%bash
cd test2

git push origin master
Everything up-to-date

Now let's fetch that work from machine #1:
In [38]:
%%bash
cd test

git pull origin master
Already up-to-date.

From https://github.com/jakevdp/test
 * branch            master     -> FETCH_HEAD

An important aside: conflict management

While git is very good at merging, if two different branches modify the same file in the same location, it simply can't decide which change should prevail. At that point, human intervention is necessary to make the decision. Git will help you by marking the location in the file that has a problem, but it's up to you to resolve the conflict. Let's see how that works by intentionally creating a conflict.
We start by creating a branch and making a change to our experiment file:
In [40]:
%%bash
cd test

git branch trouble
git checkout trouble
echo "This is going to be a problem..." >> experiment.txt
git commit -a -m"Changes in the trouble branch"
[trouble d46245e] Changes in the trouble branch
 1 file changed, 1 insertion(+)

Switched to branch 'trouble'

And now we go back to the master branch, where we change the same file:
In [41]:
%%bash
cd test

git checkout master
echo "More work on the master branch..." >> experiment.txt
git commit -a -m"Mainline work"
[master 387f3ef] Mainline work
 1 file changed, 1 insertion(+)

Switched to branch 'master'

The conflict...

So now let's see what happens if we try to merge the trouble branch into master:
In [42]:
%%bash
cd test

git merge trouble
Auto-merging experiment.txt
CONFLICT (content): Merge conflict in experiment.txt
Automatic merge failed; fix conflicts and then commit the result.

Let's see what git has put into our file:
In [43]:
%%bash
cd test

cat experiment.txt
Some crazy idea
More new content on my experiment
<<<<<<< HEAD
More work on the master branch...
=======
This is going to be a problem...
>>>>>>> trouble

At this point, we go into the file with a text editor, decide which changes to keep, and make a new commit that records our decision. I've now made the edits, in this case I decided that both pieces of text were useful, but integrated them with some changes:
In [44]:
%%bash
cd test

cat experiment.txt
Some crazy idea
More new content on my experiment
More work on the master branch...
This is going to be a problem...

Let's then make our new commit:
In [45]:
%%bash
cd test

git commit -a -m"Completed merge of trouble, fixing conflicts along the way"
git slog
[master 0213b88] Completed merge of trouble, fixing conflicts along the way
*   0213b88 Completed merge of trouble, fixing conflicts along the way
|\  
| * d46245e Changes in the trouble branch
* | 387f3ef Mainline work
|/  
* 8852251 More work, on machine #2
*   3471d04 Merge branch 'experiment'
|\  
| * 2ac278c Trying something new
* | 0c5002e The mainline keeps moving
|/  
* 110c1bb I like this new name better
* 43ed8dd I have made great progress on this critical matter.
* 609a459 This is our first commit

Merge Tools

Note: While it's a good idea to understand the basics of fixing merge conflicts by hand, in some cases you may find the use of an automated tool useful.
Git supports multiple merge tools: a merge tool is a piece of software that conforms to a basic interface and knows how to merge two files into a new one. Since these are typically graphical tools, there are various to choose from for the different operating systems, and as long as they obey a basic command structure, git can work with any of them.
4. Collaborating on github with a small team

Here we will set up a shared collaboration with one partner -- choose someone sitting next to you.
We will have two people, let's call them Alice and Bob, sharing a repository. Alice will be the owner of the repo and she will give Bob write privileges.
Find a partner & decide who will be Alice, and who will be Bob.
Note for SVN users: this is similar to the classic SVN workflow, with the distinction that commit and push are separate steps. SVN, having no local repository, commits directly to the shared central resource, so to a first approximation you can think of svn commit as being synonymous with git commit; git push.
1. Synchronization

We begin with a simple synchronization example. Working together, follow these steps:
Creating a new repository

Alice: create a new repository on github called AliceBob
Alice: create a file README.md, commit it, and push it to the remote.
Alice: on github, go to the settings for AliceBob, and add your partner to the list of collaborators
Cloning your partner's repository

Bob: clone the AliceBob repository using git clone [url]
Bob: makes changes to the README.md file and commit locally.
Bob: push changes to github.
Alice: pull Bob's changes to the local repository.
Now Alice and Bob should both have the same README.md file on their own computer.
2. Dealing with conflicts

Next, we will have both parties make non-conflicting changes each, and commit them locally. Then both try to push their changes:
Alice: create & commit a new file, alice.txt to the local repo
Bob: create & commit a new file, bob.txt to the local repo
Alice: push the latest commit to github
Bob: try to push to github. What happens?
Alice adds a new file, alice.txt to the repo and commits.
Bob adds bob.txt and commits.
Alice pushes to github.
Bob tries to push to github. What happens here?
The problem is that Bob's changes create a commit that conflicts with Alice's, so git refuses to apply them.
Bob must do
git pull origin master
And then deal with the conflict manually, then push again.
5. Full-contact github: distributed collaboration with large teams

On large teams, you don't always want all contributors to have access to the main repository. So how do you move forward? Using Pull Requests.
Again, we'll do this as an exercise with your partner:
[Brief demo of how this works on scikit-learn]
We'll practice this here, by having Alice now fork Bob's repository.
Bob: create a new repository named BobAlice with a README.md file, and push it to the github remote.
Alice: go to Bob's github page and click the fork button. You now have your own remote version of the repository, that looks like http://github.com/Alice/BobAlice.git
Alice: use git clone [url] to get a local version of your fork on your own computer.
Alice: use git remote add upstream [url] to add a pointer to Bob's remote (the original)
Alice: type git remote -v, and you should see both your own fork (called origin) and Bob's fork (called upstream)
Alice: create a new branch called alice_changes
Alice: add a file called alice.txt commit, and use git push origin alice_changes to push to the remote.
Alice: reload the github page for your own fork: there should now be a button that says "compare and pull request". Click it and fill it out.
Bob: go to your own notifications page (the blue circle in the upper-left of GitHub) and you should see a notification of Alice's Pull request. Check the diff, add some comments, and merge the changes.
Alice: on your computer, checkout the master branch, and update it from Bob's fork with git pull upstream master
Congratulations! You're now a collaborator!
This is how virtually all open source collaboration proceeds on Github!
Other useful commands

show
reflog
rebase
tag
Git resources

Introductory materials

There are lots of good tutorials and introductions for Git, which you can easily find yourself; this is just a short list of things I've found useful. For a beginner, I would recommend the following 'core' reading list, and below I mention a few extra resources:
The smallest, and in the style of this tuorial: git - the simple guide contains 'just the basics'. Very quick read.
The concise Git Reference: compact but with all the key ideas. If you only read one document, make it this one.
In my own experience, the most useful resource was Understanding Git Conceptually. Git has a reputation for being hard to use, but I have found that with a clear view of what is actually a very simple internal design, its behavior is remarkably consistent, simple and comprehensible.
For more detail, see the start of the excellent Pro Git online book, or similarly the early parts of the Git community book. Pro Git's chapters are very short and well illustrated; the community book tends to have more detail and has nice screencasts at the end of some sections.
If you are really impatient and just want a quick start, this visual git tutorial may be sufficient. It is nicely illustrated with diagrams that show what happens on the filesystem.
For windows users, an Illustrated Guide to Git on Windows is useful in that it contains also some information about handling SSH (necessary to interface with git hosted on remote servers when collaborating) as well as screenshots of the Windows interface.
Cheat sheets
Two different cheat sheets in PDF format that can be printed for frequent reference.
Beyond the basics

At some point, it will pay off to understand how git itself is built. These two documents, written in a similar spirit, are probably the most useful descriptions of the Git architecture short of diving into the actual implementation. They walk you through how you would go about building a version control system with a little story. By the end you realize that Git's model is almost an inevitable outcome of the proposed constraints:
The Git parable by Tom Preston-Werner.
Git foundations by Matthew Brett.
Git ready
A great website of posts on specific git-related topics, organized by difficulty.
QGit: an excellent Git GUI
Git ships by default with gitk and git-gui, a pair of Tk graphical clients to browse a repo and to operate in it. I personally have found qgit to be nicer and easier to use. It is available on modern linux distros, and since it is based on Qt, it should run on OSX and Windows.
Git Magic
Another book-size guide that has useful snippets.
The learning center at Github
Guides on a number of topics, some specific to github hosting but much of it of general value.
A port of the Hg book's beginning
The Mercurial book has a reputation for clarity, so Carl Worth decided to port its introductory chapter to Git. It's a nicely written intro, which is possible in good measure because of how similar the underlying models of Hg and Git ultimately are.
Intermediate tips
A set of tips that contains some very valuable nuggets, once you're past the basics.
For SVN users

If you want a bit more background on why the model of version control used by Git and Mercurial (known as distributed version control) is such a good idea, I encourage you to read this very well written post by Joel Spolsky on the topic. After that post, Joel created a very nice Mercurial tutorial, whose first page applies equally well to git and is a very good 're-education' for anyone coming from an SVN (or similar) background.
In practice, I think you are better off following Joel's advice and understanding git on its own merits instead of trying to bang SVN concepts into git shapes. But for the occasional translation from SVN to Git of a specific idiom, the Git - SVN Crash Course can be handy.
A few useful tips for common tasks

Better shell support

Adding git branch info to your bash prompt and tab completion for git commands and branches is extremely useful. I suggest you at least copy:
git-completion.bash
git-prompt.sh
You can then source both of these files in your ~/.bashrc and then set your prompt (I'll assume you named them as the originals but starting with a . at the front of the name):
source $HOME/.git-completion.bash
source$HOME/.git-prompt.sh
PS1='[\u@\h \W$(__git_ps1 " (%s)")]\$ '   # adjust this to your prompt liking
See the comments in both of those files for lots of extra functionality they offer.
Embedding Git information in LaTeX documents

(Sent by Yaroslav Halchenko) su I use a Make rule:
# Helper if interested in providing proper version tag within the manuscript
revision.tex: ../misc/revision.tex.in ../.git/index
   GITID=$$(git log -1 | grep -e '^commit' -e '^Date:' | sed  -e 's/^[^ ]* *//g' | tr '\n' ' '); \
   echo $$GITID; \
   sed -e "s/GITID/$$GITID/g" $< >|$@
in the top level Makefile.common which is included in all subdirectories which actually contain papers (hence all those ../.git). The revision.tex.in file is simply:
% Embed GIT ID revision and date
\def\revision{GITID}
The corresponding paper.pdf depends on revision.tex and includes the line \input{revision} to load up the actual revision mark.
git export

Git doesn't have a native export command, but this works just fine:
git archive --prefix=fperez.org/  master | gzip > ~/tmp/source.tgz

