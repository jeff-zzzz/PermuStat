\documentclass[10pt]{article}
\usepackage[top=1in, bottom=1in, left=1in, right=1in]{geometry}
\usepackage{hyperref, lscape, rotating, color, colortbl, amsfonts, amsmath, xcolor}  
\usepackage{graphicx}
\usepackage{verbatim}
\usepackage[round]{natbib}
\graphicspath{{./Figure/}, {./M1S1fMRIROI/}} %path for figures
\newcommand{\comments}[1]{}

\title{EAV, libMff, anaconda, RabbitMQ, pika, and GeoPy }
%\author{Jasmine Song$^{\ast}$ } 
%\address{$^{\ast}$Electrical Geodesics, Inc. 500 East 4th Ave., Suite 200, Eugene OR 97401, USA}
%\date{\today} 

\begin{document}
%\maketitle
\section{EAV and libMFF}
\begin{verbatim} 
/Shared/Users/Erik/EAV/systemPython/EAV.app.tar.gz  
/Shared/Users/Erik/libMFF/build_python/libMFF.so 
\end{verbatim} 
\begin{itemize}
\item EAV will be installed in /Applications/EAV/ 
\item Put libMFF.so in GeoPy/mff\_python/
\end{itemize}

\section{anaconda}
\begin{verbatim} 
http://continuum.io/downloads
\end{verbatim} 
\begin{itemize}
\item Download Mac OS X - 64 Bit Python 27 Graphical Installer 
\item Install Anaconda-2.2.0-MacOSX-x86\_64.pkg 
\item In .profile: will be added   
\begin{verbatim} 
# added by Anaconda 2.2.0 installer                                             
export PATH="/Users/jesong1126/anaconda/bin:$PATH" 
\end{verbatim} 
\end{itemize}

\section{RabbitMQ} \label{sec:intro} 
\subsection{Downloading and Installing RabbitMQ} 
In \href{http://www.rabbitmq.com/}{http://www.rabbitmq.com/} \\
Installation Guide \\
Mac OS X: Homebrew \\
Open a terminal.  
\begin{verbatim} 
jesong1126$ brew update 
jesong1126$ brew install rabbitmq 
==> Downloading https://homebrew.bintray.com/bottles/rabbitmq-3.4.4.mavericks.bo
######################################################################## 100.0%
==> Pouring rabbitmq-3.4.4.mavericks.bottle.tar.gz
==> Caveats
Management Plugin enabled by default at http://localhost:15672

Bash completion has been installed to:
  /usr/local/etc/bash_completion.d

To have launchd start rabbitmq at login:
    ln -sfv /usr/local/opt/rabbitmq/*.plist ~/Library/LaunchAgents
Then to load rabbitmq now:
    launchctl load ~/Library/LaunchAgents/homebrew.mxcl.rabbitmq.plist
Or, if you don't want/need launchctl, you can just run:
    rabbitmq-server
==> Summary
??  /usr/local/Cellar/rabbitmq/3.4.4: 1030 files, 27M
jasminesongspro:Cellar jesong1126$ cd /usr/local/sbin
jasminesongspro:sbin jesong1126$ ls
rabbitmq-defaults rabbitmq-plugins  rabbitmqadmin
rabbitmq-env      rabbitmq-server   rabbitmqctl
jasminesongspro:sbin jesong1126$ ./rabbitmq-server -detached
Warning: PID file not written; -detached was passed.
\end{verbatim}

\subsubsection{In .profile} 
\begin{verbatim} 
export PATH=/usr/local/sbin:$PATH
\end{verbatim}

\subsubsection{In Chrome}
\begin{verbatim} 
localhost:15672/#/
id: guest
pw: guest 
\end{verbatim}

\section{pika} \label{sec:pika} 
\subsection{python} 
\subsubsection{In ternimal }
\begin{verbatim} 
jasminesongspro:~ jesong1126$ sudo pip install pika==0.9.8 
Password:
Downloading/unpacking pika==0.9.8
  Downloading pika-0.9.8.tar.gz (56kB): 56kB downloaded
  Running setup.py (path:/private/tmp/pip_build_root/pika/setup.py) egg_info for package pika
    
Installing collected packages: pika
  Running setup.py install for pika
    
Successfully installed pika
Cleaning up...
jasminesongspro:~ jesong1126$ 
\end{verbatim}

\subsubsection{In .profile} 
\begin{verbatim} 
export PYTHONPATH=$PYTHONPATH:/usr/local/lib/python2.7/site-packages
\end{verbatim}

%send.py 
%\begin{verbatim} 
%#!/usr/bin/env python
%import pika
%
%connection = pika.BlockingConnection(pika.ConnectionParameters(
%        host='localhost'))
%channel = connection.channel()
%
%
%channel.queue_declare(queue='hello')
%
%channel.basic_publish(exchange='',
%                      routing_key='hello',
%                      body='Hello World!')
%print " [x] Sent 'Hello World!'"
%connection.close()
%\end{verbatim}
%
%receive.py 
%\begin{verbatim} 
%#!/usr/bin/env python
%import pika
%
%connection = pika.BlockingConnection(pika.ConnectionParameters(
%        host='localhost'))
%channel = connection.channel()
%
%channel.queue_declare(queue='hello')
%
%print ' [*] Waiting for messages. To exit press CTRL+C'
%
%def callback(ch, method, properties, body):
%    print " [x] Received %r" % (body,)
%
%channel.basic_consume(callback,
%                      queue='hello',
%                      no_ack=True)
%
%channel.start_consuming()
%\end{verbatim}

\subsection{anaconda ipython}  
\begin{verbatim} 
jesong1126$ conda install -c https://conda.binstar.org/auto pika
Fetching package metadata: ......
Solving package specifications: .
Package plan for installation in environment /Users/jesong1126/anaconda:

The following packages will be downloaded:

    package                    |            build
    ---------------------------|-----------------
    conda-env-2.1.4            |           py27_0          15 KB
    pika-0.9.13                |           py27_0          96 KB
    pip-6.1.1                  |           py27_0         1.4 MB
    ------------------------------------------------------------
                                           Total:         1.5 MB

The following NEW packages will be INSTALLED:

    pika:      0.9.13-py27_0

The following packages will be UPDATED:

    conda-env: 2.1.3-py27_0 --> 2.1.4-py27_0 
    pip:       6.0.8-py27_0 --> 6.1.1-py27_0 

Proceed ([y]/n)? y

Fetching packages ...
conda-env-2.1. 100% |################################| Time: 0:00:00 178.07 kB/s
pika-0.9.13-py 100% |################################| Time: 0:00:00 374.93 kB/s
pip-6.1.1-py27 100% |################################| Time: 0:00:01   1.35 MB/s
Extracting packages ...
[      COMPLETE      ]|###################################################| 100%
Unlinking packages ...
[      COMPLETE      ]|###################################################| 100%
Linking packages ...
[      COMPLETE      ]|###################################################| 100%
jesong1126$ 
\end{verbatim}

\section{GeoPy}
GeoPy\_04102015.zip


\section{Other packages} \label{sec:matplotlib} 
\begin{verbatim} 
jesong1126$ conda update conda
jesong1126$ conda update anaconda
jesong1126$ conda install pyqt 
jesong1126$ sudo pip install six
jesong1126$ sudo pip install pyparsing
jesong1126$ sudo pip install matplotlib
\end{verbatim}





%% -------------------------------------------------------------------------
%\bibliographystyle{IEEEbib}
%\bibliography{ISBIreference}
%\bibliography{strings,refs}

\end{document}
